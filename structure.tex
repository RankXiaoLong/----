 %%%%%%%%%%%%%%%%%%%%%%%%%%%%%%%%%%%%%%%%%% 
 % @File    : c:\Users\Administrator\Desktop\Econometrics\structure.tex
 % @Date    : 2021-02-11 15:14:30
 % @Author  : RankFan
 % @Email   : 1917703489@qq.com
 % -----
 % Last Modified: 2021-02-15 11:25:53
 % Modified By: Rank_fan
 % -----
 %%%%%%%%%%%%%%%%%%%%%%%%%%%%%%%%%%%%%%%%%% 

\usepackage{ctex}
\usepackage{type1cm}
\usepackage{ragged2e}

%====================================== 标题设置=============== ================================%
\usepackage{caption}  % 标题
\usepackage{subfig} % 子图
\usepackage{authblk} % 作者
\usepackage{tocbibind} % 目录
\usepackage[Glenn]{fncychap} % 章节封面设置

%====================================== 章节设置=============== ================================%
\usepackage{titlesec} % 可对标题的式样进行全面精细地设置。
\usepackage{indentfirst} % 设置缩进的距离
\usepackage[left=2cm,right=2cm,top=2.5cm,bottom=3cm]{geometry} % 页面边距

%====================================== 表格、图设置=============== ================================%

%\usepackage{color}
% \usepackage{tikz,pgfplots} % tikz绘制图
%\usepackage{mathpazo}

\usepackage{enumerate} % enumerate 环境
\usepackage{fancyref} % 交叉引用
\usepackage{graphicx} % 插入图片
\usepackage{algorithm} % 算法
\usepackage{bm} % 对公式字母进行加粗
\usepackage{listings} % 在LaTex中添加代码
\usepackage{mathrsfs} % 插入花体公式
\usepackage{array} % 矩阵
\usepackage{multirow,booktabs} % 表格多行
\usepackage{arraycols} % 设置array的列间距
\usepackage{threeparttable}
\usepackage{longtable}

%====================================== 数学公式设置=============== ================================%
% \usepackage{tocvsec2} % 标题目录由固定层次深度改为可变层次深度
% \usepackage[utf8]{inputenc} % 用来嵌入字体
% \usepackage{calc}

\usepackage{amsmath,amssymb} % 数学公式
\usepackage{datetime} % 时间
\usepackage{pdfpages} % 就是将PDF文件加入到封面位置
% \usepackage[colorlinks,citecolor=blue,linkcolor=blue,urlcolor=red,bookmarks, bookmarksnumbered]{hyperref} % hyperref 参数 breaklinks,
\usepackage{cases,empheq} % 大括号公式,有色公式 http://blog.sina.com.cn/s/blog_5e16f1770100h5gs.html
\usepackage{hyperref}
\hypersetup{ colorlinks,citecolor=blue,linkcolor=blue,urlcolor=red,
			bookmarks, bookmarksnumbered,
			unicode = true }


%=====================================================================================%
\setlength{\parindent}{2em} %===================首行缩进2字符

%================================== tikz ================================%
% \usetikzlibrary{arrows,shapes,intersections,angles,calc,quotes,through,decorations.text}
% \usetikzlibrary{shapes.geometric, arrows}

%================================== 标题设置 ================================%
\titlelabel{{\bf \S} ~ \thetitle \quad}

%================================== 自定义样式设置 ================================%
\lstdefinestyle{mystyle}{
	basicstyle=%
	\ttfamily
	\lst@ifdisplaystyle\small\fi
}

%================================== 设置中文字体 ================================%
\xeCJKsetup{AutoFakeBold=true}
\setCJKmainfont{FZShuSong-Z01}[BoldFont={FZHei-B01},ItalicFont={FZKai-Z03}]
\setCJKsansfont{FZHei-B01}
\setCJKmonofont{FZShuSong-Z01}
% \setsansfont{Verdana}

%================================== 设置英文字体 ================================%TeX Gyre Pagella
\setmainfont[Mapping=tex-text]{Times New Roman}%--------------------------------英文衬线字体
% \tikzset{elegant/.style={smooth,thick,samples=50,cyan}}
% \tikzset{eaxis/.style={->,>=stealth}}
% \definecolor[named]{color01}{rgb}{.2,.4,.6} % Color used in the title page

%================================== 盒子 ================================%TeX Gyre Pagella
% \newlength{\lablen}
% \setlength{\lablen}{0pt}
% \settowidth{\lablen}{2021 年 2 月 8 日}

%================================== newcommand ================================%
\newcommand{\topcaption}{%
		\setlength{abovecaptionskip}{0pt} % 标题前
		\setlength{belowcaptionskip}{2em} % 标题后
		\caption}
\newcommand*{\hei}{\CJKfamily{heiti}} 
\newcommand*{\num}{pi}
\newcommand{\thickhline}{%
	\noalign {\ifnum 0=`}\fi \hrule height 1pt
	\futurelet \reserved@a \@xhline}
\newcolumntype{"}{@{\hskip\tabcolsep\vrule width 1pt\hskip\tabcolsep}}
\newcommand*{\email}[1]{%
\normalsize\href{mailto:#1}{#1}\par
}

%===================================renewcommand ==============================================%
\renewcommand {\thetable} {\thechapter{}.\arabic{table}}
\renewcommand {\thefigure} {\thesection{}.\arabic{figure}}
\renewcommand\chaptermark[1]{\markboth{Chapter \thechapter\quad #1}{}}
\renewcommand\sectionmark[1]{\markright{\thesection\quad #1}}
\renewcommand{\today}{\number\year 年 \number\month 月 \number\day 日}
\renewcommand*{\Affilfont}{ \it }
\renewcommand\Authands{ and  } 
\setcounter{secnumdepth}{3}

%===================================设置中文章节标题==========================================================%
\renewcommand\sectionmark[1]{%
 \markright{\CTEXifname{\CTEXthesection——}{}#1}}

%===================================newtheorem ==========================================================%
\newtheorem{theorem}{$ \bf{{Theorem}}  $}[section]
\newtheorem{mydef}{Definition}[chapter]
\newtheorem{myproof}{Proof}[chapter]
\newtheorem{myexample}{Example}[chapter]
\newtheorem{mypty}{Property}[chapter]
\newtheorem{mycor}{Corollary}[chapter]

% =================================设置节标题==========================================================%
\numberwithin{equation}{section}

% ================================表格caption设置==========================================================%
\captionsetup[table]{%
	labelsep = quad ,  % title与font之间的距离
	name = 表 %表的名字
}
% ====================================================%