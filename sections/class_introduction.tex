\chapter*{课程简介}

\section*{课程简介}
\noindent \textbf{课程号:}

\noindent \textbf{课程名称(含英文主):}研究生《中级计量经济学》(Intermediate Econometrics)

\noindent \textbf{学分:2}

\noindent \textbf{周学时:4}

\noindent \textbf{预修课程:}经济学、高等数学、概率论和数理统计

\noindent \textbf{内容简介:}

首先介绍计量经济学中必须具备的数学知识如高等代数中矩阵、概率论与数理统计中点估计、有效估计、一致估计、区间估计、假设检验、大样本与极限理论等。而后介绍古典线性回归模型、多元线性回归模型、带有线性约束的多元线性回归模型及其假设检验、正态线性统计模型的最大似然估计、古典线性的大样本理论、非球形扰动与广义最小二乘、异方差性、非线性回归模型等。

\noindent \textbf{选用教材或参考书:}

\noindent \textbf{教材:}William H. Greene, \href{https://bbs.pinggu.org/a-503441.html}{Econometrics Analysis}, fourth edition。

\noindent \textbf{参考书:}

\noindent 1.William H. Greene,经济计量分析,Econometrics Analysis 的翻译, 中国社会科学出版社。

\noindent 2.课件。 
	
\section*{教学大纲}
\noindent \textbf{课程号:}

\noindent \textbf{课程名称(含英文主):}研究生《中级计量经济学》(Intermediate Econometrics)

\noindent \textbf{学分:2}

\noindent \textbf{周学时:4}

\noindent \textbf{预修课程:}经济学、高等数学、概率论和数理统计

\noindent \textbf{一、课程的教学目的和基本要求:}

本课程为已具备经济学,概率论和数理统计以及初级计量经济学的研究生开设的《中级计量经济学》。目的是为他们今后在经济和金融领域能够独立开展科学研究和调查提供坚实的统计与计量经济学的方法与技巧。本课程的重点是使学生充分掌握和理解以下三个方面的知识与技能:

\noindent 1.计量经济学的理论与原理;

\noindent 2.计量经济学中广泛使用的统计推断知识,方法与技巧;

\noindent 3.掌握各种模型需要的条件与模型的局限性和适用性。

\noindent \textbf{二、课程内容与学时分配:}

\noindent 第一章 \ 引言
$ \cdots\cdots\cdots\cdots\cdots\cdots\cdots\cdots\cdots\cdots\cdots\cdots\cdots\cdots\cdots
\cdots\cdots\cdots\cdots\cdots\cdots\cdots\cdots\cdots\cdots\cdots \text{1学时} $
\begin{itemize}
	\setlength{\itemsep}{-2pt} 
	\item 计量经济学概念
	\item 为什么学计量经济学
	\item 计量经济学模型 
\end{itemize}	

\noindent 第二章 \ 矩阵的基础知识	
$ \cdots\cdots\cdots\cdots\cdots\cdots\cdots\cdots\cdots\cdots\cdots\cdots\cdots\cdots\cdots
\cdots\cdots\cdots\cdots\cdots\cdots\cdots\cdots\cdots \text{4学时} $
\begin{itemize}
	\setlength{\itemsep}{-2pt}
	\item 矩阵的概念与运算
	\item 矩阵的特征根与特征向量
	\item 矩阵的二次型与二项式
	\item 矩阵的微分
\end{itemize}

\noindent 第三章 \ 概率论与数理统计
$ \cdots\cdots\cdots\cdots\cdots\cdots\cdots\cdots\cdots\cdots\cdots\cdots\cdots\cdots\cdots
\cdots\cdots\cdots\cdots\cdots\cdots\cdots\cdots \text{4学时} $
\begin{itemize}
	\setlength{\itemsep}{-2pt}
	\item 随机变量与概率
	\item 分布函数与中心极限定理 
	\item 二元态分布与多元正态分布
	\item 样本与样本的分布函数
	\item 统计量及其分布
	\item 点估计
	\item 有效估计
	\item 一致估计
	\item 区间估计
	\item 假设检验
\end{itemize}

\noindent 第四章 \ 古典线性回归模型
$ \cdots\cdots\cdots\cdots\cdots\cdots\cdots\cdots\cdots\cdots\cdots\cdots\cdots\cdots\cdots
\cdots\cdots\cdots\cdots\cdots\cdots\cdots\cdots\text{4学时} $
\begin{itemize}
	\setlength{\itemsep}{-2pt}
	\item 古典线性回归模型与其假设条件
	\item 最小二乘回归
	\item 方差分析
	\item 最小二乘统计量的有限样本性质
	\item 预测
\end{itemize}	

\noindent 第五章 \ 多元线性回归模型
$ \cdots\cdots\cdots\cdots\cdots\cdots\cdots\cdots\cdots\cdots\cdots\cdots\cdots\cdots\cdots
\cdots\cdots\cdots\cdots\cdots\cdots\cdots\cdots \text{4学时} $
\begin{itemize}
	\setlength{\itemsep}{-2pt}	
	\item 多元线性回归模型与其假设条件
	\item 最小二乘回归
	\item 方差分析
	\item 最小二乘统计量的有限样本性质
	\item 预测	
\end{itemize}	
	
\noindent 第六章 \ 带有线性约束的多元线性回归模型及其假设检验
$ \cdots\cdots\cdots\cdots\cdots\cdots\cdots\cdots\cdots\cdots\cdots\cdots\cdots
 \text{4学时} $
\begin{itemize}
	\setlength{\itemsep}{-2pt}	
	\item 带有线性约束的多元线性回归模型与其假设条件
	\item 线性约束的检验
	\item 参数带有约束的最小二乘回归
	\item Wald检验
	\item 实例	
\end{itemize}	

\noindent 第七章 \ 正态线性统计模型的最大似然估计
$ \cdots\cdots\cdots\cdots\cdots\cdots\cdots\cdots\cdots\cdots\cdots\cdots\cdots\cdots\cdots
\cdots\cdots\cdots \text{4学时} $
\begin{itemize}
	\setlength{\itemsep}{-2pt}	
	\item 模型及其假设条件
	\item 模型求解
	\item 与最小二乘估计量的比较	
\end{itemize}		
	
\noindent 第八章 \ 古典线性回归的大样本理论
$ \cdots\cdots\cdots\cdots\cdots\cdots\cdots\cdots\cdots\cdots\cdots\cdots\cdots\cdots\cdots
\cdots\cdots\cdots\cdots\cdots \text{4学时} $
\begin{itemize}
	\setlength{\itemsep}{-2pt}
	\item 最小二乘统计量的有限样本性质
	\item 古典回归模型的渐近分布理论
	\item 最小二乘估计量的渐近正态性
	\item 标准检验统计量的渐近行为
\end{itemize}	
	
\noindent 第九章 \ 非线性回归模型
$ \cdots\cdots\cdots\cdots\cdots\cdots\cdots\cdots\cdots\cdots\cdots\cdots\cdots\cdots\cdots
\cdots\cdots\cdots\cdots\cdots\cdots\cdots\cdots\cdots \text{2学时} $
\begin{itemize}
	\setlength{\itemsep}{-2pt}
	\item 非线性回归模型
	\item 可供选择的几个统计量
	\item 假设检验与参数约束
	\item Box-Cox 变换
\end{itemize}		
	
\noindent 第十章 \ 异方差性
$ \cdots\cdots\cdots\cdots\cdots\cdots\cdots\cdots\cdots\cdots\cdots\cdots\cdots\cdots\cdots
\cdots\cdots\cdots\cdots\cdots\cdots\cdots\cdots\cdots\cdots\cdots \text{2学时} $
\begin{itemize}
	\setlength{\itemsep}{-2pt}
	\item OLS估计的探讨
	\item 异方差性的检验
	\item GLS估计
	\item 二阶段估计
\end{itemize}			