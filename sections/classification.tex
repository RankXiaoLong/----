\chapter{计量经济模型的类别}

\section{模型的类别}
一般的模型是广义回归模型,即假设
$$ y=F(\boldsymbol{\beta}, \boldsymbol{X})+\boldsymbol{\varepsilon}, \quad \mathbb{E} [\boldsymbol{\varepsilon}]=\boldsymbol{0}, \quad \mathbb{E}\left[\boldsymbol{\varepsilon \varepsilon}^{\prime}\right]=\sigma^{2} \boldsymbol{\Omega} $$
其中$ \boldsymbol{\Omega} $是一般的正定矩阵,$ \sigma^{2}\boldsymbol{\Omega} $是样本的协方差矩阵。
	
{\bf{假设}}$ \mathrm{Cov}\left(\varepsilon_{i}, \varepsilon_{j}\right)
=\rho_{ij} $,样本的协方差矩阵 $  \boldsymbol{\Sigma } $ (the covariance matrix)是:
\begin{equation*}
	\boldsymbol{\Sigma} = \left(\begin{array}{cccc}
		\sigma_{1}^{2} & \rho_{12} & \cdots & \rho_{1 n} \\
		\rho_{21} & \sigma_{2}^{2} & & \vdots \\
		\vdots & & \ddots & \rho_{n-1, n} \\
		\rho_{n 1} & \cdots & \rho_{n, n-1} & \sigma_{n}^{2}
	\end{array}\right)
\end{equation*}

$ \sigma^{2}\boldsymbol{\Omega} $中应该有$ 1+2+\dots+n=\frac{1}{2}n(n+1) $未知的参数,再加上未知参数$ \boldsymbol{\beta} $的个数,是一个只有$ n $个样本点难以完成任务的,即使完成,效率和准确性是不高的。即不简化模型我们将一事无成。
	
\noindent \textbf{模型1 Large-sample Model}
$$ \boldsymbol{\Sigma} = \mathbb{E}\left(\boldsymbol{\varepsilon \varepsilon}^{\prime}\right)=\sigma^{2}\boldsymbol{I} $$
	
\noindent \textbf{模型2 异方差(Heteroscedasticity)}
\begin{equation*}
	\boldsymbol{\Sigma} = \left(\begin{array}{cccc}
		\sigma_{1}^{2} & 0 & \cdots & 0 \\
		0 & \sigma_{2}^{2} & & \vdots \\
		\vdots & & \ddots & 0 \\
		0 & \cdots & 0 & \sigma_{n}^{2}
	\end{array}\right)
\end{equation*}

即使这样,也有超过$ n $个未知的参数要估计,所以,进一步假设组间异方差(group-wise)

\begin{equation*}
	\boldsymbol{\Sigma} = \left(\begin{array}{cccccccccc}
		\sigma_{1}^{2} & & & & & & & & & \\
		& \ddots & & & & & & & & \\
		& & \sigma_{1}^{2} & & & & & & & \\
		& & & \sigma_{2}^{2} & & & & & & \\
		& & & & \ddots & & & & & \\
		& & & & & \sigma_{2}^{2} & & & & \\
		& & & & & & \ddots & & & \\
		& & & & & & & \sigma_{g}^{2} & & \\
		& & & & & & & & \ddots & \\
		& & & & & & & & & \sigma_{g}^{2}
	\end{array}\right)
\end{equation*}

\noindent \textbf{模型3 自相关(Autocorrelation)}
\begin{equation*}
	\boldsymbol{\Sigma} = \left(\begin{array}{ccccc}
		\sigma^{2} & \rho & \rho^{2} & \cdots & \rho^{n-1} \\
		\rho & \sigma^{2} & \rho & & \rho^{n-2} \\
		\rho^{2} & \rho & \sigma^{2} & & \vdots \\
		\vdots & & & \ddots & \rho \\
		\rho^{n-1} & \cdots & & \rho & \sigma^{2}
	\end{array}\right)
\end{equation*}

此时我们需要估计两个参数$ \left(\sigma^{2}, \rho\right) $。
	
\noindent \textbf{模型4 ARCH(条件异方差)或GARCH(广义条件异方差)}

\begin{equation*}
	\boldsymbol{\Sigma} = \left(\begin{array}{cccc}
		\sigma_{1}^{2} & 0 & \cdots & 0 \\
		0 & \sigma_{2}^{2} & & \vdots \\
		\vdots & & \ddots & 0 \\
		0 & \cdots & 0 & \sigma_{n}^{2}
	\end{array}\right)
\end{equation*}

对于不同的观测值$ \sigma_{i}^{2} $是不同的,但是它们彼此间也存在一定关系,如:
\begin{align*}
	&\text{ARCH}: \sigma_{k}^{2}=a+b \sigma_{k-1}^{2}\text{在条件} \mathrm{Cov}\left(\varepsilon_{i}, \varepsilon_{j}\right)
	=0\text{下} \\	
	&\text{GARCH}: \sigma_{k}^{2}=a+b \sigma_{k-1}^{2}+c \sigma_{k-2}^{2}+\cdots 
\end{align*}

\section{事例:社会保障水平与国内生产总值}

直现上看,社会保障水平的相关因素中,最主要的因素是人均国内生产总值。只有人均国内生产总值的增长,才会有资金支撑社会保障的各项支出,我们可以建立相应的线性回归模型:$ Y=a+bX+\varepsilon $
	
利用有关国家的数据,算出常数项$ a $和系数$ b $,如下:
	
社会保障水平与人均GDP增长之间的相关函数和回归方程:
\begin{table}[htb!]
	\centering
	\setlength{\tabcolsep}{2em}
	\begin{threeparttable}[b]
		\caption{社会保障水平与人均GDP增长之间的相关函数和回归方程} 
		\begin{tabular}{cclc}
			\hline
			国家 & 相关系数 & 回归方程$Y=a+bX$      & 样本年份             \\
			\hline
			英国 & 0.956 & $Y=14.1+0.0034X$   & \multirow{6}{*}{1960—1995} \\
			瑞典 & 0.964 & $Y=10.68+0.0064X$  &                            \\
			丹麦 & 0.940 & $Y=10.14+0.0056X$  &                            \\
			美国 & 0.903 & $Y=10.46+0.00034X$ &                            \\
			日本 & 0.988 & $Y=7.62+0.00078X$  &                            \\
			德国 & 0.947 & $Y=16.37+0.00081X$ &                            \\
			\hline
		\end{tabular}
		\begin{tablenotes} \tiny
			\item[1] 世界银行,世界发展报告(1982—1998)北京:中国财政经济出版社
			\item[2] 联合国,人类发展报告(1982—1999)伦敦:天津大学出版社
		\end{tablenotes}
	\end{threeparttable}
\end{table}
	
	
	
从统计分析结果证明了2点:
	
1、社会保障水平与人均GDP队长之间存在着高度相关。(相关系数在0.94至0.98之间)
	
2、回归方程中的自变量系数$ b $值,福利型国家明显都高于自保公助型国家,上述关系表明,人均GDP每增长一亿本币,社会保障支出相应增长,福利型国家为$ 0.003\% \sim 0.006\% $,自保公助型国家为$ 0.0003\% \sim 0.0008\% $,二者相差一个小数点,从而说明,在相同人均国内生产总值增长速度下,福利型国家社会保障水平的上升速度快于自保公助型国家。
		
\textbf{问题:从样本到母体的的检验还没有完成!}